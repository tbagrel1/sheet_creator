\documentclass[a4paper, 11pt,oneside, fleqn]{article}

% Import des packages
\usepackage[latin1]{inputenc}
\usepackage[T1]{fontenc}
\usepackage[frenchb]{babel}
\usepackage{tabulary}
\usepackage[top=2cm, bottom=2cm, left=2cm, right=1cm]{geometry}
\usepackage{setspace}
\usepackage{hyperref}
\usepackage{frcursive}
\usepackage{collcell}
\usepackage{bookman}

% Debut du document
\begin{document}

% Creationdes nouvelles commandes
\newcommand{\x}{\times}
\newcolumntype{h}{>{\Large\bfseries\arraybackslash}C}
\newcolumntype{g}{>{\large\collectcell\MakeUppercase}h<{\endcollectcell}}
\newcolumntype{i}{>{\Large\cursive}h}
\renewcommand{\arraystretch}{1.5}

% Parametres du document
\sloppy
\pagestyle{empty}
\begin{onehalfspace}

% Corps du document

    % En-tete :
\sffamily \noindent \Large Pr\'enom : \fbox{\begin{minipage}{9cm} \vspace{1.2cm}\hspace{9cm} \end{minipage}} \hspace{1.5cm}\Large Date :\vspace{2mm}\\
\begin{minipage}{12cm}
\begin{center}
\Large\textbf{Langage \'ecrit - Le mot \MakeUppercase{cochon}}
\end{center}
\normalsize Nous avons d\'ecouvert puis appris \`a m\'emoriser le mot\\
\MakeUppercase{cochon}\\
Nous pouvons le retrouver dans une liste de mots : \end{minipage}\\
\vspace{0.25cm}\\
 
    % Tableaux : 
        % Capitale
\large\noindent En capitale :
\begin{center}
{\huge \textbf{\MakeUppercase{cochon}}}
\vspace{0.25cm}\\
\begin{tabulary}{17cm}{|g|g|g|g|g|}
\hline
royaume & distance & comble & chapeau & cochon \\
\hline
tristement & choeur & cochon & avenir & courber \\
\hline
houille & charger & alcool & cochon & spacieux \\
\hline
conseil & sacoche & chocolat & chance & cochon \\
\hline
�pargner & tant�t & s�curit� & cochon & sacoche \\
\hline
\end{tabulary}
\end{center}
\vspace{0.5cm}

% Fin du document
\end{onehalfspace}
\end{document}
