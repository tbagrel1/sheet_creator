\documentclass[a4paper, 11pt,oneside, fleqn]{article}

% Import des packages
\usepackage[utf8]{inputenc}
\usepackage[T1]{fontenc}
\usepackage[frenchb]{babel}
\usepackage{tabulary}
\usepackage[top=2cm, bottom=2cm, left=2cm, right=1cm]{geometry}
\usepackage{setspace}
\usepackage{hyperref}
\usepackage{frcursive}
\usepackage{collcell}
\usepackage{bookman}

% Debut du document
\begin{document}

% Creationdes nouvelles commandes
\newcommand{\x}{\times}
\newcolumntype{h}{>{\Large\bfseries\arraybackslash}C}
\newcolumntype{g}{>{\large\collectcell\MakeUppercase}h<{\endcollectcell}}
\newcolumntype{i}{>{\Large\cursive}h}
\renewcommand{\arraystretch}{1.5}

% Parametres du document
\sloppy
\pagestyle{empty}
\begin{onehalfspace}

% Corps du document

    % En-tete :
\sffamily \noindent \Large Pr\'enom : \fbox{\begin{minipage}{9cm} \vspace{1.2cm}\hspace{9cm} \end{minipage}} \hspace{1.5cm}\Large Date :\vspace{2mm}\\
\begin{minipage}{12cm}
\begin{center}
\Large\textbf{Langage \'ecrit - Le mot \MakeUppercase{english}}
\end{center}
\normalsize Nous avons d\'ecouvert puis appris \`a m\'emoriser le mot\\
\MakeUppercase{english}\\
Nous pouvons le retrouver dans une liste de mots : \end{minipage}\\
\vspace{0.25cm}\\
 
    % Tableaux : 
        % Capitale
\large\noindent En capitale :
\begin{center}
{\huge \textbf{\MakeUppercase{english}}}
\vspace{0.25cm}\\
\begin{tabulary}{17cm}{|g|g|g|g|g|}
\hline
gant & english & noix & indigne & gens \\
\hline
redire & english & hier & entre & peinture \\
\hline
matinal & reflet & english & poitrine & client \\
\hline
english & boiteux & cruel & multiple & géranium \\
\hline
rouler & exercice & angoisse & english & semaine \\
\hline
\end{tabulary}
\end{center}
\vspace{0.5cm}

        % Script
\large\noindent En script :
\begin{center} {\huge \textbf{english}}
\vspace{0.25cm}\\
\begin{tabulary}{17cm}{|h|h|h|h|h|}
\hline
bien & jeunesse & english & rayonner & relire \\
\hline
sonner & bientôt & english & gamin & miette \\
\hline
légume & brin & retrousser & faible & parvenir \\
\hline
gêner & aiguille & tenter & puisque & obtenir \\
\hline
english & poulet & vingt & peintre & dépendre \\
\hline
\end{tabulary}
\end{center}
\vspace{0.5cm}

        % Cursive
\large\noindent En cursive :
\begin{center}
{\huge \textbf {{\cursive english}}}
\vspace{0.25cm}\\
\begin{tabulary}{17cm}{|i|i|i|i|i|}
\hline
ceux & onduler & oranger & english & inerte \\
\hline
limite & souvent & ci-joint & imiter & argenter \\
\hline
exister & english & soulier & teinte & arranger \\
\hline
fillette & english & viande & cuiller & replier \\
\hline
taper & encourir & english & résonner & enchanté \\
\hline
\end{tabulary}
\end{center}

% Fin du document
\end{onehalfspace}
\end{document}
