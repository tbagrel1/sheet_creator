\documentclass[a4paper, 11pt,oneside, fleqn]{article}

% Import des packages
\usepackage[utf8]{inputenc}
\usepackage[T1]{fontenc}
\usepackage[frenchb]{babel}
\usepackage{tabulary}
\usepackage[top=2cm, bottom=2cm, left=2cm, right=1cm]{geometry}
\usepackage{setspace}
\usepackage{hyperref}
\usepackage{frcursive}
\usepackage{collcell}
\usepackage{bookman}

% Debut du document
\begin{document}

% Creationdes nouvelles commandes
\newcommand{\x}{\times}
\newcolumntype{h}{>{\Large\bfseries\arraybackslash}C}
\newcolumntype{g}{>{\large\collectcell\MakeUppercase}h<{\endcollectcell}}
\newcolumntype{i}{>{\Large\cursive}h}
\renewcommand{\arraystretch}{1.5}

% Parametres du document
\sloppy
\pagestyle{empty}
\begin{onehalfspace}

% Corps du document

    % En-tete :
\sffamily \noindent \Large Pr\'enom : \fbox{\begin{minipage}{9cm} \vspace{1.2cm}\hspace{9cm} \end{minipage}} \hspace{1.5cm}\Large Date :\vspace{2mm}\\
\begin{minipage}{12cm}
\begin{center}
\Large\textbf{Langage \'ecrit - Le mot \MakeUppercase{jambon}}
\end{center}
\normalsize Nous avons d\'ecouvert puis appris \`a m\'emoriser le mot\\
\MakeUppercase{jambon}\\
Nous pouvons le retrouver dans une liste de mots : \end{minipage}\\
\vspace{0.25cm}\\
 
    % Tableaux : 
        % Capitale
\large\noindent En capitale :
\begin{center}
{\huge \textbf{\MakeUppercase{jambon}}}
\vspace{0.25cm}\\
\begin{tabulary}{17cm}{|g|g|g|g|g|}
\hline
matures & bastidon & bonasse & sabotez & jambon \\
\hline
glanâmes & embarrez & macabre & timbras & jambon \\
\hline
carburer & charron & enlignasse & jambon & embellie \\
\hline
grogneries & jambon & jackets & herbacés & mollasse \\
\hline
matriçât & détonaient & boutâtes & jambon & blatère \\
\hline
\end{tabulary}
\end{center}
\vspace{0.5cm}

        % Script
\large\noindent En script :
\begin{center} {\huge \textbf{jambon}}
\vspace{0.25cm}\\
\begin{tabulary}{17cm}{|h|h|h|h|h|}
\hline
cambre & calaison & oignon & enjambes & jambon \\
\hline
mulsion & galeries & acompte & herbagerez & jambon \\
\hline
jablent & adhésions & mosaïste & jambon & rajahs \\
\hline
gominant & bruirai & emboîter & ébroueriez & surexposeras \\
\hline
bitumais & jambon & bossâmes & jardiner & impétrai \\
\hline
\end{tabulary}
\end{center}
\vspace{0.5cm}

        % Cursive
\large\noindent En cursive :
\begin{center}
{\huge \textbf {{\cursive jambon}}}
\vspace{0.25cm}\\
\begin{tabulary}{17cm}{|i|i|i|i|i|}
\hline
marquées & jardines & inobservable & fût-on & jambon \\
\hline
tinamous & baisâmes & jambon & ajustées & égare \\
\hline
jaunîmes & jambon & bordures & grimas & embuai \\
\hline
resquillasses & dépoétisaient & broderai & jaugeait & jambon \\
\hline
emperlas & javeline & cambrons & jambon & lombards \\
\hline
\end{tabulary}
\end{center}

% Fin du document
\end{onehalfspace}
\end{document}
