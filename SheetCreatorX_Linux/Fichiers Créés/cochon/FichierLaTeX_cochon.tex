\documentclass[a4paper, 11pt,oneside, fleqn]{article}

% Import des packages
\usepackage[utf8]{inputenc}
\usepackage[T1]{fontenc}
\usepackage[frenchb]{babel}
\usepackage{tabulary}
\usepackage[top=2cm, bottom=2cm, left=2cm, right=1cm]{geometry}
\usepackage{setspace}
\usepackage{hyperref}
\usepackage{frcursive}
\usepackage{collcell}
\usepackage{bookman}

% Debut du document
\begin{document}

% Creationdes nouvelles commandes
\newcommand{\x}{\times}
\newcolumntype{h}{>{\Large\bfseries\arraybackslash}C}
\newcolumntype{g}{>{\large\collectcell\MakeUppercase}h<{\endcollectcell}}
\newcolumntype{i}{>{\Large\cursive}h}
\renewcommand{\arraystretch}{1.5}

% Parametres du document
\sloppy
\pagestyle{empty}
\begin{onehalfspace}

% Corps du document

    % En-tete :
\sffamily \noindent \Large Pr\'enom : \fbox{\begin{minipage}{9cm} \vspace{1.2cm}\hspace{9cm} \end{minipage}} \hspace{1.5cm}\Large Date :\vspace{2mm}\\
\begin{minipage}{12cm}
\begin{center}
\Large\textbf{Langage \'ecrit - Le mot \MakeUppercase{cochon}}
\end{center}
\normalsize Nous avons d\'ecouvert puis appris \`a m\'emoriser le mot\\
\MakeUppercase{cochon}\\
Nous pouvons le retrouver dans une liste de mots : \end{minipage}\\
\vspace{0.25cm}\\
 
    % Tableaux : 
        % Capitale
\large\noindent En capitale :
\begin{center}
{\huge \textbf{\MakeUppercase{cochon}}}
\vspace{0.25cm}\\
\begin{tabulary}{17cm}{|g|g|g|g|g|}
\hline
chêne & docteur & prochain & filer & cochon \\
\hline
approche & alcool & clocher & prochain & simple \\
\hline
poche & choisir & occasionner & chaise & cochon \\
\hline
mouchoir & consoler & honneur & brebis & démarche \\
\hline
mâchoire & cochon & choix & clocher & poche \\
\hline
\end{tabulary}
\end{center}
\vspace{0.5cm}

        % Script
\large\noindent En script :
\begin{center} {\huge \textbf{cochon}}
\vspace{0.25cm}\\
\begin{tabulary}{17cm}{|h|h|h|h|h|}
\hline
coureur & prochain & varier & mouchoir & couche \\
\hline
cochon & marchand & considérer & école & colonial \\
\hline
péché & hypocrite & reproche & brochure & coeur \\
\hline
chou & cochon & mouchoir & arracher & complet \\
\hline
poche & cochon & reproche & corne & champ \\
\hline
\end{tabulary}
\end{center}
\vspace{0.5cm}

        % Cursive
\large\noindent En cursive :
\begin{center}
{\huge \textbf {{\cursive cochon}}}
\vspace{0.25cm}\\
\begin{tabulary}{17cm}{|i|i|i|i|i|}
\hline
cochon & couler & chêne & consoler & chose \\
\hline
beaucoup & pratiquer & conduite & bouton & couleur \\
\hline
chez & choc & cochon & reproche & acheteur \\
\hline
mars & cochon & choisir & chaise & neveu \\
\hline
méthode & cochon & féroce & enfermer & prochain \\
\hline
\end{tabulary}
\end{center}

% Fin du document
\end{onehalfspace}
\end{document}
